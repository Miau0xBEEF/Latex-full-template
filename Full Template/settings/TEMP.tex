
 
 


 
 
 


%nötig da sonst pgf nicht mit tikz zusammenarbeitet
\usepackage{etex}
\usepackage{indentfirst}




%für die Erstellung von Tabellen automatisch gelesen aus csv daten
\usepackage{lipsum}
\usepackage{pdflscape}
\usepackage{pdfpages}
\usepackage{lscape}
%%%%\usepackage[square,numbers]{natbib}


%\usepackage{textcomp}
%\usepackage{ucs}

%\captionsetup[lstlisting]{font={small,tt}}


%TODO wichtig, ohne geht nicht
\usepackage{pstricks}
%für alle tex versionen
%\usepackage{auto-pst-pdf}
%\usepackage{siunitx}

\usepackage[most]{tcolorbox}
\usepackage{makecell}













%   %   %   %   %   %   %   %   %   %   % Biblography  %   %   %   %   %   %   %   %   %   %   %   %   %   %   %   %   
\begin{filecontents}{Myelectronicmanual.dbx}
\DeclareDatamodelEntrytypes{Myelectronicmanual}
\DeclareDatamodelFields[type=list, datatype=name]{partdescription}
\DeclareDatamodelEntryfields[Myelectronicmanual]{
part,
partdescription,
url,
year}
\end{filecontents}








  
  

















%==========================================================================
% MISC COMMANDS
%==========================================================================




%~~~~~~~~~~~~~~~ 					Wichtig 							 ~~~~~~~~~~~~~~~
%~~~~~~~~~~~~~~~ C:\Program Files\MiKTeX2.9\scripts\xindy\xindy.pl 	 ~~~~~~~~~~~~~~~
%~~~~~~~~~~~~~~~ replace in line 779 								 ~~~~~~~~~~~~~~~
%~~~~~~~~~~~~~~~ @styles = glob("$lang_dir/$variant$cp*-lang.xdy"); 	 ~~~~~~~~~~~~~~~
%~~~~~~~~~~~~~~~ @styles = glob("'$lang_dir/$variant$cp*-lang.xdy'");	 ~~~~~~~~~~~~~~~
%~~~~~~~~~~~~~~~ https://tex.stackexchange.com/questions/251221/miktex-and-xindy-problems ~~~~~~~~~~~~~~~

\usepackage[xindy]{imakeidx}
%\usepackage[nomain,acronym,xindy,toc,automake]{glossaries} % nomain, if you define glossaries in a file, and you use \include{INP-00-glossary}
\usepackage[
nonumberlist,                       % keine Seitenzahlen anzeigen
acronym,                            % ein Abkürzungsverzeichnis erstellen
%toc,                                % Einträge im Inhaltsverzeichnis
section]                            % im Inhaltsverzeichnis auf section-Ebene erscheinen
{glossaries}     

\GlsSetXdyCodePage{duden-utf8}
\newglossary[slg]{siemens}{sgr}{sgo}{Siemens Glossary}

                 






\pgfplotsset{compat=1.15} 






















%==========================================================================
% START OF DOCUMENT
%==========================================================================

\begin{document}





\input{sections/main/title.tex}
\input{sections/main/declaration_of_honesty.tex}


\tableofcontents 

%===============================
% BEGIN CHAPTER LinuxCNC
%===============================







%=========================================================
% APPENDIX
%=========================================================
\newpage
\section{Appendix}

\addcontentsline{toc}{subsection}{Literaturverzeichnis}

\printbibliography[title={Literatur},  type=book]

\printbibliography[title={Artikel},  type=article]
\newpage


\printbibliography[title={Manuals}, type=manual, nottype=electronic, nottype=book, nottype=article]
%
%
\newpage 
\printbibliography[title={online}, type=online]







\lstset
{ 
    basicstyle=\footnotesize,
    numbers=left,
    stepnumber=1,
    showstringspaces=false,
    tabsize=1,
    breaklines=true,
    breakatwhitespace=false,
    backgroundcolor=\color{WhiteBackground},
}

\addcontentsline{toc}{subsection}{Anhang}
\label{sec: Anhang}



\printindex
\appendix 






\newpage
\addcontentsline{toc}{subsection}{Glossar-Allgemein}
\printglossary[title=Glossar-Allgemein]


\newpage
\addcontentsline{toc}{subsection}{Glossar-Siemens}
\printglossary[type=siemens, title=Glossar-Siemens]


\newpage
\addcontentsline{toc}{subsection}{Abkürzungsverzeichnis}
\printglossary[type=\acronymtype ,title=Abkürzungsverzeichnis]

\newpage
\addcontentsline{toc}{subsection}{Nomenclature}
\printnomenclature
\clearpage

\end{document}

