\documentclass[a4paper,12pt]{article}
 
\usepackage[german]{babel}	% Deutsche überschriften und Silbentrennung
\usepackage[T1]{fontenc}
\usepackage[utf8]{inputenc}

\usepackage{fancyhdr} %%kopf fusszeile
\usepackage{datetime} % Uhrzeit 

\usepackage[space]{grffile}
\parindent 0.0cm %Absatz
\parskip 1,5ex plus 0.5ex minus 0.5ex %%abstand zw zwei absaetzen
\usepackage{setspace} %Zeilenabstände
\onehalfspacing 

\usepackage{geometry}
\geometry{a4paper, left=40mm, right=30mm, bottom=25mm, top=25mm}




\usepackage[xindy]{imakeidx}
%\usepackage[nomain,acronym,xindy,toc,automake]{glossaries} % nomain, if you define glossaries in a file, and you use \include{INP-00-glossary}
\usepackage[
nonumberlist,                       % keine Seitenzahlen anzeigen
acronym,                            % ein Abkürzungsverzeichnis erstellen
%toc,                                % Einträge im Inhaltsverzeichnis
section]                            % im Inhaltsverzeichnis auf section-Ebene erscheinen
{glossaries}     

\GlsSetXdyCodePage{duden-utf8}
\newglossary[slg]{siemens}{sgr}{sgo}{Siemens Glossary}

\usepackage{translator}
                 
\makeglossaries 
\makeindex



%%%%%%%%%%%%%%% CUSTOM COMMANDS 

\newcommand{\graytt}[1]{\textcolor{gray}{\texttt{\normalfont #1}}}
 