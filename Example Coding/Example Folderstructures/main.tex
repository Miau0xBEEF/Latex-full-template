\documentclass[10pt,a4paper]{article}
\usepackage[utf8]{inputenc}
\usepackage{amsmath}
\usepackage{amsfonts}
\usepackage{amssymb}
\usepackage{graphicx}
\author{Claas-Focke Schmidt}
\title{Folder Structures }

% % % % % % % % % % % % PROJECT RELATED PACKAGES % % % % % % % % % % % % % %

%%%%%% Dirtreee to print folder -structure %%%%%%
\usepackage{dirtree}  %Darstellung eines Dateibaumes oder Verzeichnisstruktur 

%%%%%% Using fontawesome for icons %%%%%%
\usepackage{fontawesome5}  

%%%%%% xcolor for colors %%%%%% 
\usepackage{xcolor}

\begin{document}

\title{Folder structures with dirtree}


\section{Simple dirtree}
This is a simple dirtree: 
\dirtree{%
.1 MyProject.
.2 build.
.2 subfolder.
.3 subsubfolder.
.3 subsubfolder.
.3 subsubfolder.
}


\section{dirtree in figure}
Figure \ref{fig:SmpldirInFigure} shows a simple dirtree in a figure envirorment. Don`t forget to use make command to use \\ref\{\}
\begin{figure}[h]
\dirtree{%
.1 MyProject.
.2 build.
.2 subfolder.
.3 subsubfolder.
.3 subsubfolder.
.3 subsubfolder.
}
\caption{Simple dirtree in figure-envirorment}
\label{fig:SmpldirInFigure}
\end{figure}


\section{dirtree with Colors and Icons}
The figure \ref{fig:moreComplexDir} shows a dirtree with colored items. 
\begin{figure}[h]
\dirtree{%
.1 \textcolor{red}{\faFolder\ MyProject}.
.2 \textcolor{red}{\faFolder\ build}.
.2 \faFile\ CMakeLists.txt.
.2 \faFile\ main.c.
.2 \faFile\ myServer.h.
.2 \faFile\ myServer.c.
}
\caption{Structure with icons and colored structures}
\label{fig:moreComplexDir}
\end{figure}

\end{document}