\documentclass{article}
\usepackage[utf8]{inputenc}

\usepackage{biblatex}% Necesarry because biber is configured as backend


%Einfuegen von Grafiken 
\usepackage{graphicx}
\graphicspath{ {./pic/}{./pic/nested} } %Sollte erweitert werden 
\usepackage{float} 	% H = Here bei figures
%\usepackage{subfigure} %Outdated
\usepackage{subcaption} %Newer version for subfigure. 
\usepackage[usenames,dvipsnames]{color}



\newcommand{\insertfigure}[4]{
  \begin{figure}[H]
    \centering
    \includegraphics[width=#2]{#1}
    \caption{#3}
    \label{#4}
  \end{figure}
}


%%%%%%%%%%%%%%%%%%%%%%%%%%%%%%%%%%%%%%%%%%%%%%%%%%%%%%%%%%%%%%%%%%%%%%%%%%%
%%%%%%%%%%%%%%%%%%%%%%%%% Begin of Document %%%%%%%%%%%%%%%%%%%%%%%%%%%%%%%
%%%%%%%%%%%%%%%%%%%%%%%%%%%%%%%%%%%%%%%%%%%%%%%%%%%%%%%%%%%%%%%%%%%%%%%%%%%

\begin{document}
\textbf{Including Images to a document}\\

There are multiple ways of insertig images. Here are three examples for including a graphic to the document. Als here is the figure-envirorment used. You can also include images by includegraphics. 
\begin{itemize}
\item[1] Inserting by a custom function: Image \ref{fig:Example_from_insert}
\item[2] Inserting by standard includegraphics with full relative path: Image \ref{fig:Example_from_include}
\item[3] Inserting by standard includegraphics path added to graphicspath: Image \ref{fig:Nested_Image}
\end{itemize}

\textbf{1. Function}\\
This is convenient because it does not take much space in the document. You can create one function for all images. On top it keeps the consistency to all images. Less convenient is that the auto-reference does not recognizes the label: \textit{fig:Example\_from\_insert}
\insertfigure{pic/example.png}{1.0\textwidth}{Example image from insertfiguter}{fig:Example_from_insert}

\textbf{2. Full description}\\
This methode takes much space and some time to add images. But you will have full controll about the image. But you have to add every image with full configuration. 
\begin{figure}[H]
  \centering
    \includegraphics[width=1.0\textwidth]{pic/example.png}
    \caption{Example from IncludeGraphics}
    \label{fig:Example_from_include}
\end{figure}


\textbf{3. Full description}\\
Using the graphicspath in the preamble adds a little bit compfotability to the document. You just have to know the filename and to add the folder that contains the data you want to include. In this example the image is nested in a subfolder that was added to graphicspath. 
\begin{figure}[h]
\centering
\includegraphics[width=1\textwidth]{nastyExample.png}
\caption{Nested image in subdirectory}
\label{fig:Nested_Image}
\end{figure}


\subsection{Side by Side Figure}

\begin{figure}
\centering
\begin{subfigure}{.5\textwidth}
  \centering
  \includegraphics[width=.4\linewidth]{nastyExample}
  \caption{A subfigure}
  \label{fig:sub1}
\end{subfigure}%
\begin{subfigure}{.5\textwidth}
  \centering
  \includegraphics[width=.4\linewidth]{nastyExample}
  \caption{A subfigure}
  \label{fig:sub2}
\end{subfigure}
\caption{A figure with two subfigures}
\label{fig:test}
\end{figure}

\end{document}
